\section{Conclusion}
\label{conclusion-and-future-work}

Programmable storage is a viable method for eliminating duplication of complex
error-prone software used as workarounds for storage system deficiencies. We
propose that systems expose their services in a safe way allowing application
developers to customize system behavior to meet their needs while not
sacrificing correctness. To illustrate the benefits of this approach we
presented Malacology, a programmable storage system that facilitates the
construction of new services by re-purposing existing subsystem abstractions of
the storage stack. For information on Malacology, Mantle, and ZLog, visit our
website at \href{programmability.us}{programmability.us}.

\oldcommentone{Future work will focus on constructing more interfaces to
support a wide variety of storage system services that can be configured
on-the-fly in existing systems. This work is one point along that path to
producing general-purpose storage systems that can target  special-purpose
applications.  Ultimately we want to utilize declarative methods for expressing
new services.}

\section{Acknowledgements}

We thank the EuroSys reviewers for their hard work, attentiveness, and
genuinely helpful suggestions. We also thank Mahesh Balakrishnan for
championing and shepherding the paper. This work was partially funded by the
Center for Research in Open Source Software
(\href{cross.ucsc.edu}{cross.ucsc.edu}), the DOE Award DE-SC0016074, and the
NSF Award 1450488.
